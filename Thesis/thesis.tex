\documentclass[12pt]{aghdpl}
% \documentclass[language=en,11pt]{aghdpl}  % praca w języku angielskim

%---------------------------------------------------------------------------

\author{Bartłomiej Kręgielewski}
\shortauthor{B. Kręgielewski}

%\titlePL{Przygotowanie bardzo długiej i pasjonującej pracy dyplomowej w~systemie~\LaTeX}
%\titleEN{Preparation of a very long and fascinating bachelor or master thesis in \LaTeX}

\titlePL{Aplikacja webowa do kolejkowania, uruchamiania i analizy symulacji}
\titleEN{Web app for queueing, running and analyzing simulations}


%\shorttitlePL{Przygotowanie pracy dyplomowej w~systemie \LaTeX} % skrócona wersja tytułu jeśli jest bardzo długi
%\shorttitleEN{Preparation of a long and fascinating thesis in \LaTeX}


% Dopuszczalne wartości[1,2]:
% * "Projekt dyplomowy" - na koniec studiów I stopnia
% * "Praca dyplomowa" - na koniec studiów II stopnia
% [1] Zasady dyplomowania w roku akademickim 2020/2021 (Decyzja Dziekana WEAIiIB nr 16/2020 z dnia 9 grudnia 2020 roku)
% [2] Załącznik nr 1a) do Decyzji nr 16/2020 Dziekana Wydziału EAIiIB z dnia 09 grudnia 2020 r.
\thesistype{Praca dyplomowa}
%\thesistype{Master of Science Thesis}

\supervisor{dr hab. inż. Jarosław Wąs}
%\supervisor{Jarosław Wąs, PhD}

\degreeprogramme{Automatyka i Robotyka}
%\degreeprogramme{Automatics and Robotics}

\date{2021}

%\department{Katedra Informatyki Stosowanej}
%\department{Department of Applied Computer Science}

\faculty{Wydział Elektrotechniki, Automatyki, Informatyki i Inżynierii Biomedycznej}
%\faculty{Faculty of Electrical Engineering, Automatics, Computer Science and Biomedical Engineering}

% TODO
\acknowledgements{Serdecznie dziękuję \dots tu ciąg dalszych podziękowań np. dla promotora, żony, sąsiada itp.}


\begin{document}

\titlepages
\RedefinePlainStyle

\setcounter{tocdepth}{2}
\tableofcontents
\clearpage

% TODO
%\chapter{Wprowadzenie}
\label{cha:wprowadzenie}

\LaTeX~jest systemem składu umożliwiającym tworzenie dowolnego typu dokumentów (w~szczególności naukowych i technicznych) o wysokiej jakości typograficznej (\cite{Dil00}, \cite{Lam92}). Wysoka jakość składu jest niezależna od rozmiaru dokumentu -- zaczynając od krótkich listów do bardzo grubych książek. \LaTeX~automatyzuje wiele prac związanych ze składaniem dokumentów np.: referencje, cytowania, generowanie spisów (treśli, rysunków, symboli itp.) itd.

\LaTeX~jest zestawem instrukcji umożliwiających autorom skład i wydruk ich prac na najwyższym poziomie typograficznym. Do formatowania dokumentu \LaTeX~stosuje \TeX a (wymiawamy 'tech' -- greckie litery $\tau$, $\epsilon$, $\chi$). Korzystając z~systemu składu \LaTeX~mamy za zadanie przygotować jedynie tekst źródłowy, cały ciężar składania, formatowania dokumentu przejmuje na siebie system.

%---------------------------------------------------------------------------

\section{Cele pracy}
\label{sec:celePracy}


Celem poniższej pracy jest zapoznanie studentów z systemem \LaTeX~w zakresie umożliwiającym im samodzielne, profesjonalne złożenie pracy dyplomowej w systemie \LaTeX.

\subsection{Jakiś tytuł}

\subsubsection{Jakiś tytuł w subsubsection}


\subsection{Jakiś tytuł 2}

%---------------------------------------------------------------------------

\section{Zawartość pracy}
\label{sec:zawartoscPracy}

W rozdziale~\ref{cha:pierwszyDokument} przedstawiono podstawowe informacje dotyczące struktury dokumentów w \LaTeX u. Alvis~\cite{Alvis2011} jest językiem\textellipsis

Jeszcze kilka odnośników do bibliografii \cite{PeDa04,BuDo03}.

% itd.
% \appendix
% \include{dodatekA}
% \include{dodatekB}
% itd.

\printbibliography

\end{document}
