\documentclass[12pt]{aghdpl}
\newcommand{\dotnet}{.NET}
\newcommand{\docker}{Docker}
% Use and configure packages
\usepackage{graphicx}
\graphicspath{{./Assets/}}

\usepackage{float}
% \documentclass[language=en,11pt]{aghdpl}  % praca w języku angielskim

%---------------------------------------------------------------------------

\author{Bartłomiej Kręgielewski}
\shortauthor{B. Kręgielewski}

\titlePL{Aplikacja webowa do kolejkowania, uruchamiania i analizy symulacji}
\titleEN{Web app for queueing, running and analyzing simulations}

\shorttitlePL{Aplikacja webowa do kolejkowania, uruchamiania i analizy symulacji} % skrócona wersja tytułu jeśli jest bardzo długi
\shorttitleEN{Web app for queueing, running and analyzing simulations}


% Dopuszczalne wartości[1,2]:
% * "Projekt dyplomowy" - na koniec studiów I stopnia
% * "Praca dyplomowa" - na koniec studiów II stopnia
% [1] Zasady dyplomowania w roku akademickim 2020/2021 (Decyzja Dziekana WEAIiIB nr 16/2020 z dnia 9 grudnia 2020 roku)
% [2] Załącznik nr 1a) do Decyzji nr 16/2020 Dziekana Wydziału EAIiIB z dnia 09 grudnia 2020 r.
\thesistype{Praca dyplomowa}
%\thesistype{Master of Science Thesis}

\supervisor{dr hab. inż., prof. AGH Jarosław Wąs}
%\supervisor{Jarosław Wąs, PhD}

\degreeprogramme{Informatyka}
%\degreeprogramme{Computer Science}

\date{2022}

%\department{Katedra Informatyki Stosowanej}
%\department{Department of Applied Computer Science}

\faculty{Wydział Elektrotechniki, Automatyki, Informatyki i Inżynierii Biomedycznej}
%\faculty{Faculty of Electrical Engineering, Automatics, Computer Science and Biomedical Engineering}

\acknowledgements{Serdecznie dziękuję dr hab. inż., prof. AGH Jarosławowi Wąsowi za objęcie mojej pracy dyplomowej swoją opieką oraz wsparcie w procesie jej tworzenia. Dziękuję również moim rodzicom i rodzeństwu za nieustanne wsparcie oraz koleżankom i kolegom z studiów, za ich opinie i rady.}

\begin{document}

\titlepages
\RedefinePlainStyle

\setcounter{tocdepth}{2}
\tableofcontents
\clearpage

\chapter{Wstęp}
\label{cha:wstep}

\par Symulacje komputerowe służą nam do rozmaitych celów. Od prostych symulacji mających na celu przedstawienie działania jakiegoś zjawiska do naprawdę zaawansowanych, wieloagentowych systemów, mających na celu przewidzenie zachowania tłumu podczas opuszczania obiektu sportowego po wieczornym meczu siatkówki. Informacje uzyskiwane w ten sposób pomagają inżynierom na całym świecie projektować lepsze systemy, znajdować niedopatrzenia w już istniejących, jak i je optymalizować. Jednak aby osiągnąć te cele potrzebna jest znacząca moc obliczeniowa, którą często nie dysponujemy, używając komputerów osobistych. Nawet jeżeli szybkość z jaką urządzenia, które znajdują się pod lub na naszych biurkach, potrafią dokonywać obliczeń rosła przez ostatnie dziesięciolecia zgodnie z Prawem Moore'a,\cite{mollick2006establishing} to tak samo rosły nasze oczekiwania wobec nich. Kiedy dziś mówimy o optymalizacji procesów, symulowaniu zjawiska fizycznego czy wspomnianej wcześniej symulacji środowiska składającego się z wielu agentów, to chcemy przeprowadzić nie jeden czy dwa eksperymenty, ale setki lub nawet tysiące.

\par Problem ten pozwala nam rozwiązać tak zwany \emph{Cloud Computing}. Oferuje on możliwość oddelegowania obliczeń do specjalnie przeznaczonego serwera (bądź też serwerów). Takie rozwiązania istnieją już w sieci, jednak jeżeli zaczniemy się im przyglądać, to dostrzeżemy, że większość z nich skłania się ku symulacjom konkretnych zjawisk fizycznych po dostarczeniu pliku w odpowiednim formacie. Jest to dość istotne ograniczenie, szczególnie kiedy interesują nas symulacje innej natury. Oczywiście trzeba wspomnieć również o superkomputerach, których reprezentantem jest między innymi znajdujący się w Cyfronecie AGH \emph{\prometheusAgh{}}. Maszyny te są jednak trudno dostępne. Uruchomienie na nich jakikolwiek obliczeń wymaga zazwyczaj otrzymania specjalnej zgody (grantu w przypadku \emph{\prometheusAgh{}}a), jak i samodzielnego przygotowania pewnego rodzaju systemu, który uruchamiałby automatycznie przygotowaną przez na symulację.

\par W mojej pracy inżynierskiej poruszam przedstawione powyżej problemy oraz proponuję ich rozwiązanie.

%---------------------------------------------------------------------------

\section{Zakres pracy}
\label{sec:zakresPracy}

Zakresem pracy jest:
\begin{itemize}
	\item Przygotowanie standardu do implementacji symulacji.
	\item Stworzenie aplikacji webowej, spełniającej następujące warunki:
	      \begin{itemize}
		      \item Istnienie interfejsu, dzięki któremu możliwa jest obsługa zarówno przez użytkownika, jak i przez aplikacje \emph{3\textsuperscript{rd}-party}.
		      \item Istnienie systemu obsługi błędów.
		      \item Obsługa wspomnianego wcześniej standardu symulacji (opisanego dokładniej w sekcji \ref{sec:simulationStandard}).
		      \item Możliwość bezpiecznego uruchomienia kodu \emph{3\textsuperscript{rd}-party}.
		      \item Przechowywanie informacji o użytkowniku, plikach przez niego dostarczonych, jak i parametrach i rezultatach przeprowadzonych symulacji.
	      \end{itemize}
	\item Przetestowanie działania stworzonego systemu.
\end{itemize}

%---------------------------------------------------------------------------

\section{Zawartość pracy}
\label{sec:zawartoscPracy}

Zawartością pracy jest:
\begin{itemize}
	\item Określenie celu i motywacji projektu.
	\item Uchwycenie problemu od strony technicznej, wskazanie miejsc, które mogą okazać się trudne w implementacji.
	\item Opisanie zastosowanych technologii oraz wzorców projektowych.
	\item Prezentacja oraz wyjaśnienie działania aplikacji powstałej w toku pracy nad projektem.
	\item Sformułowanie podsumowania zawierającego uzyskane rezultaty oraz pomysły na dalszy rozwój stworzonego systemu.
\end{itemize}
\chapter{Cel i motywacja}
\label{cha:celIMotywacja}

\par Zaprojektowanie i stworzenie modelu symulacji to tylko pierwszy krok. W zależności od natury konkretnego modelu, należy znaleźć odpowiednie wartości parametrów albo od razu uruchomić system z wybranymi uprzednio parametrami. Dopiero po tym procesie można przejść do głównego celu, dla którego w większości zaczął powstawać dany projekt, czyli do wnioskowania. Złudnym mogłoby być wrażenie zakończenia prac na etapie finalizacji implementacji. To właśnie testowanie i prawidłowe działanie symulacji potrafi zająć najwięcej czasu. Często niestety tego czasu brakuje, szczególnie w kontekście podejmowania różnej aktywności. Nie chcemy, aby nasze główne narzędzie pracy było zajęte przez obliczenia, które mogą trwać godzinami, a w skrajnych przypadkach nawet całymi dniami.

\par W trakcie okresu studiowania wraz z koleżankami i kolegami z kierunku, zaimplementowaliśmy wiele różnych symulacji. W większości były to symulacje niedeterministyczne, na bazie których wnioskowanie jest możliwe dopiero po wielokrotnym ich uruchomieniu. W niektórych przypadkach oznaczałoby wiele godzin pracy naszych komputerów, na co niestety nie mogliśmy sobie pozwolić. Skorzystanie z zewnętrznego serwisu oferującego udostępnienie komputerów również wydawało się kłopotliwe w aranżacji, a nawet gdyby udałoby się nam uzyskać dostęp do takowych maszyn to pojawiłaby się potrzeba samodzielnej implementacji dodatkowego systemu kolejkowania zadań ich i gromadzenia rezultatów.

\par Właśnie w celu przeciwstawienia się tym trudnościom powstał pomysł na niniejszy projekt. Sprowadzenia procesu uruchomiania systemu z odpowiednimi parametrami, jak i kolekcjonowania oraz prezentowania danych uzyskanych w postaci rezultatów do czynności trywialnej, dzięki stworzonemu systemowi informatycznemu. Głównym celem jest uproszczenie tego procesu, jak i odciążenie komputera osobistego osoby pracującej nad danym modelem.
\chapter{Przedstawienie problemu od strony technicznej}
\label{cha:przedstawienieProblemuOdStronyTechnicznej}
\chapter{Zastosowane technologie}
\label{cha:zastosowaneTechnologie}

\par W celu rozwiązania problemów przedstawionych w rozdziale \ref{cha:przedstawienieProblemuOdStronyTechnicznej} postanowiono zastosować następujące technologie.

\section{\dotnet{}}

\par Platforma \dotnet{} jako darmowa, \emph{open-source}owa oraz nieustannie rozwijana, wydaje się naturalnym wyborem dla stworzenia \emph{Back-End}u proponowanej aplikacji. Posiada ona uznane przez developerów \emph{framework}i, które pozwalają na szybkie tworzenie skalowalnych aplikacji webowych. Ważnym jej atutem jest zdolność do dynamicznego ładowania i usuwania z pamięci tak zwanych \emph{assemblies}, czyli fundamentalnych bloków, które zawierają informacje o klasach i metodach, oraz ich implementacje. Pozwala to na uruchomienie zewnętrznego kodu bez potrzeby rekompilacji aplikacji.

\par Wersja jaka została wybrana to \emph{\dotnet{} 6}, czyli najnowsza dostępna (zapewniająca najdłuższe wsparcie LTS \footnote{Long-term Support}\footnote{Wsparcie dla \dotnet{} 6 ma trwać do Listopada 2024 roku.}), obsługująca język \verb|C#| w wersji 10. Ważnym atutem tej edycji .NET jest jej \emph{corss-platform}owość. Platforma ta od wersji \dotnet{} Core (która ukazała się w 2016 roku) wspiera wszystkie najważniejsze \emph{OS}\footnote{Operating System}, czyli \emph{Windows}a, \emph{Linux}a i \emph{MacOS}a, a wliczając \emph{Xamarin}\footnote{Implementacja standardu \dotnet{} na platformach mobilnych.} również platformy mobilne (\emph{iOS} i \emph{Android}). Pozwala to na uruchamianie aplikacji, korzystając z serwerów działających w oparciu o system operacyjny \emph{Linux}, jak i w lekkich kontenerach opartych również o niego\cite{DOTNET_DOCUMENTATION}.

\subsection{ASP.NET Core}

\par Popularny, \emph{open-source}owy \emph{framework} do tworzenia aplikacji webowych w \dotnet{}. Został on wybrany ze względu na istniejące do niego rozszerzenia i przystosowanie do bycia uruchamianym na wielu platformach, w tym przy użyciu \emph{Docker}a\cite{ASPNET_DOCUMENTATION}.

\subsection{Entity Framework Core}

\par \emph{Cross-platform}owy \emph{framework} ułatwiający zarządzanie danymi zapisanymi w bazie danych. Pozwala on na automatyczne mapowanie obiektów i ich właściwości\english{Properties} do odpowiadającej reprezentacji w formie rekordu, jak i na tworzenie relacji między tabelami. Dodana w ten sposób dodatkowa warstwa abstrakcji pozwala na skorzystanie z dowolnego \emph{provider}a spośród wspieranych. Dodatkowo \emph{Entity Framework Core} wspiera tak zwany \emph{Lazy Loading}, co pozwala znacząco przyśpieszyć działanie aplikacji poprzez ładowanie tylko wymaganych danych i późniejsze doładowywanie dodatkowych\cite{ENTITY_FRAMEWORK_DOCUMENTATION}.

\subsection{Swagger}
\label{subsec:swagger}

\par \emph{Swagger} jest narzędziem do generowania dokumentacji i interfejsu, który pozwala zagłębić się w tę dokumentację, jak i przykładowych danych wejściowych i wyjściowych. Narzędzie to zostało wybrane, aby usprawnić proces dokumentowania stworzonej aplikacji. Dodatkowym atutem jest możliwość integracji z \emph{Netonsoft JSON}, czyli inną biblioteką wykorzystaną w tym projekcie\cite{SWAGGER_DOCUMENTATION}.

\subsection{Docker.Dotnet}

\par \emph{Open-source}owa biblioteka, która jest rozwijana pod banerem \emph{.NET Foundation}\footnote{Organizacja mająca na celu rozwój oprogramowania \emph{open-source} dla platformy \emph{\dotnet{}}.}. Pozwala ona na wykonywanie zapytań do \emph{Docker Engine API} w sposób asynchroniczny. Możliwe jest dzięki niej budowanie obrazów, tworzenie, uruchamianie i usuwanie kontenerów oraz sprawdzanie ich stanu. Istotną jej zaletą jest wsparcie dla wykonywania operacji w sposób asynchroniczny\cite{DockerDotNet_GitHub}.

\subsection{Pozostałe biblioteki}

\begin{itemize}
	\item MediatR - prosta biblioteka implementująca wzorzec projektowy \emph{Mediator}. Wspiera ona \emph{request/response}, polecenia\english{Commands}, \emph{query}, powiadomienia\english{Notifications} oraz wydarzenia\english{Events}. Dodatkowo może działać zarówno synchronicznie, jak i asynchronicznie. Więcej informacji na temat tego wzorca, można znaleźć w sekcji \ref{subsec:zastosowaneWzorceProjektowe}.
	\item AutoMapper - biblioteka pozwalająca na automatyczne mapowanie właściwości obiektów. Jej główną zaletą jest możliwość szybkiego tworzenia nowych instancji (na przykład DTO\footnote{Data Transfer Object}), bez konieczności samodzielnej implementacji generycznych fragmentów kodu.
	\item FluentValidation - weryfikacja dostarczonych danych jest koniecznością podczas tworzenia oprogramowania, w interakcję z którym, wchodzi użytkownik. Istotnym elementem jest, aby takowa weryfikacja była projektowana, według pewnego wzorca i przystępna do modyfikacji w przyszłości. Problemy te rozwiązuje biblioteka \emph{FluentValidation}. Dostarcza ona narzędzia, które pozwalają tworzyć zestawy reguł, na bazie których nastąpi później, na przykład weryfikacja zapytania.
	\item Newtonsoft JSON - biblioteka zawierająca narzędzia wspomagające działanie na obiektach zapisanych w formacie \emph{JSON}. Zapewnia ona funkcjonalność potrzebną do serializacji, jak i deserializacji z dodatkową możliwością przechowywania informacji o typie danych. Pozwala to na dokładne odtworzenie konkretnych instancji.
\end{itemize}

\section{\docker{}}

\par Platforma dostosowana do przygotowywania oraz uruchamiania oprogramowania w formie tak zwanych kontenerów. Wykorzystuje do tego \emph{OS-level virtualization}, co pozwala na osiągnięcie lepszej wydajności i mniejsze zużycie pamięci operacyjnej systemu \emph{host}a, niż na przykład tradycyjna wirtualizacja z użyciem \emph{Virtual Machine}\cite{DOCKER_DOCUMENTATION}.

\section{PostgreSQL}

\par Darmowy i \emph{open-source}owy system zarządzania relacyjnymi bazami danych. Posiada oficjalny, wspierany obraz \emph{\docker}a.

\section{Postman API Platform}

\par Aplikacja służąca do testowania API. Pozwala na wykonywanie zapytań, edytowanie ich treści, łączenie ich w grupy. Wspiera wiele rodzaj uwierzytelniania, w tym z użyciem \emph{Token}u. Dodatkowo istnieje możliwość definiowania zmiennych, które można wykorzystać w \emph{request}ach, tworzenia skryptów, jak i testów\cite{POSTMAN_DOCUMENTATION}.

\section{Git}

\par Jako system kontroli wersji zostało wybrane oprogramowanie \emph{Git}. Narzędzie to jest \emph{open-source}owe oraz ekstensywnie stosowane w współczesnym rozwoju oprogramowania zarówno przez hobbistów, jak i profesjonalistów. Pomimo jednoosobowej natury pracy dyplomowej, która nie powodowała konieczności synchronizacji wersji projektu z resztą zespołu, możliwość śledzenia zmian, jak i rozwijania oraz testowania wielu funkcjonalności równolegle (dzięki możliwości skorzystania z tak zwanych gałęzi\english{Branches}) pozostaje nieoceniona.
\chapter{Realizacja rozwiązania}
\label{cha:realizacjaRozwiazania}

\par Aby rozwiązać problem przedstawiony w rozdziale \ref{cha:celIMotywacja} postanowiono wykorzystać technologie \dotnet{} i \docker{}. Serwer działa w oparciu o \emph{ASP.NET Core} i \emph{Entity Framework Core}, oraz może zostać uruchomiony zarówno bezpośrednio, jak i w formie kontenera (więcej informacji znajduje się w sekcji \ref{sec:wymaganiaIKonfiguracja}).

\par Komunikacja z aplikacją odbywa się za pomocą protokołu \texttt{\https{}}. \emph{API} przyjmuje i zwraca odpowiedzi w formacie \emph{JSON}\footnote{JavaScript Object Notation}.

\section{Architektura}

\par Odpowiedni podział odpowiedzialności i ustalenie komunikacji między komponentami było jednym z najważniejszych elementów całego projektu. Każdy z komponentów powinien spełniać jasno określone zadania, oraz pewną abstrakcję aby w przypadku późniejszych modyfikacji dało się je łatwo zastąpić. Z tą myślą powstał projekt systemu. Inspiracją dla niego był

\begin{figure}[H]
	\includegraphics[width=\linewidth]{Component Communication Diagram}
	\caption{Diagram komunikacji między komponentami}
	\label{fig:componentCommunicationDiagram}
\end{figure}

\par Na załączonym powyżej diagramie \ref{fig:componentCommunicationDiagram}, widzimy w jaki sposób komunikują się pomiędzy sobą poszczególne komponenty. Użytkownik wchodzi w interakcję tylko z warstwą \emph{API}, która zajmuje się tylko i wyłącznie obsługą zapytań. Całość logiki biznesowej została oddelegowana do warstwy aplikacji. Ta też warstwa (a konkretnie jej komponent \emph{Docker Container Manger}) jest odpowiedzialna za uruchamianie i zarządzanie kontenerami \emph{\docker{}}a.

\par Za obsługę bazy danych odpowiedzialna jest warstwa \emph{Persistence}. Wszystkie aktywności wymagające przechowania nowych rekordów, aktualizacji lub usunięcia już istniejących są przez nią obsługiwane.

\par \emph{Docker Host} czyli system, na którym jest uruchomiony \emph{\docker{}}. Komunikacja z nim następuje poprzez \emph{Docker Engine API}. Ważnym do zaznaczenia tutaj jest fakt, że może to być albo ten sam \emph{host}, na którym jest uruchomiona nasz aplikacja, ale niekoniecznie musi. Możliwe jest połączanie się z \emph{remote host}em przez internet co prezentuje możliwość na proste poprawienie skalowalności aplikacji w przyszłość. Obecnie jednak założono, że będzie to ten sam \emph{host}, na którym działa aplikacja, co umożliwiło uproszczenie wymiany informacji z kontenerami, która odbywa się poprzez system plików.

\section{Implementacja}

\subsection{Struktura Solucji}

\par W \emph{\dotnet{}} istnieje koncept tak zwanych \emph{solucji}\english{Solutions}. Mają one na celu zgromadzenie w sobie projektów, które razem tworzą rozwiązanie danego zagadnienia. Podejście to pozwala na wyrazisty podział kodu, jak i na wielokrotne jego użycie, zgodne z zasadą \emph{DRY}\footnote{Don't Repeat Yourself}. Odpowiednia zaplanowanie naszego rozwiązania potrafi znacząco przyśpieszyć pracę.

\begin{figure}[H]
	\includegraphics[width=\linewidth]{Solution Structure Diagram}
	\caption{Diagram Struktury Solucji}
	\label{fig:solutionStructureDiagram}
\end{figure}

\par \emph{Docker-Compose} zawiera wszystkie potrzebne informacje do uruchomienia całości rozwiązania, w formie kontenerów \emph{\docker{}}. Umożliwia to uruchomienie aplikacji nie posiadając zainstalowanej platformy \emph{\dotnet{}}, jako iż wszystkie wymagane komponenty zostaną automatycznie pobrane i dodane do odpowiedniego obrazu \emph{\docker{}}.

\par Za punkt wejściowy do naszej aplikacji uznajemy projekt \emph{BackendAPI}. Odpowiada on za uruchomienie i skonfigurowanie wszystkich serwisów. W nim również znajdują się wszystkie kontrolery, logika związana z uwierzytelnianiem użytkownika, jak i pliki konfiguracyjne. \emph{Endpoint}y, które się tutaj znajdują nie posiadają w sobie żadnej logiki biznesowej. Ich zadaniem jest stworzenie odpowiedniego \emph{request}u dla aplikacji, która to dopiero zajmie się jego obsłużeniem.

\par \emph{Application} jest głównym projektem, który spina wszystkie pozostałe. Znajduje się w nim cała logika biznesowa. Odpowiada on za między innymi tworzenie użytkowników i ich autoryzację, zbieranie parametrów i wyników symulacji oraz obsługę pozostałych zapytań, w tym tych związanych z kontenerami (więcej informacji na ten temat znajduje się w sekcjach \ref{sec:dockerContainerManager} i \ref{sec:dockerWatchService}).

\par Odpowiedzialność związana z obsługą symulacji dostarczonych przez użytkowników należy do \emph{SimulationHandler}. Znajdziemy tu interfejs \texttt{ISimulationHandler} i jego implementację. Klasa ta potrafi tworzyć obiekty \emph{SimulationStandard}, w tym przeprowadzać ich serializację i deserializację. Kolejną istotną cechą jest dynamiczne ładowanie i rozładowanie tak zwanych \emph{Assemblies}, jak i sprawdzanie ich poprawność i zgodności z \emph{SimulationStandard}.

\par Projekty \emph{Persistence} i \emph{Domain} odpowiadają za przechowywanie danych. Działają one w oparciu o \emph{Entity Framework Core}. Pierwszy z nich odpowiada za rozszerzenie tak zwanego \emph{DbContext}\footnote{Specjalna klasa pochodząca z Entity Framework Core definiująca pewien kontekst danych.}, w którym znajdują się deklaracje kolekcji, które będą mapowane do poszczególnych tabel w bazie danych. Znajdują się tutaj również migracje, wspomagające ciągły rozwój aplikacji, poprzez zapewnianie mechanizmu do prostego aktualizowania przechowywanych danych w wypadku zmiany ich struktury. Projekt \emph{Domain} reprezentuje z kolej poszczególne byty na których operuje nasza aplikacja. Są one bezpośrednio mapowane do odpowiednich rekordów. Warstwa ta została zaprojektowana przy pomocy tak zwanego podejścia \emph{Code First}. Oznacza to, że odpowiednie struktury w bazie danych zostały automatycznie wygenerowane na podstawie zdefiniowanych klas.

\par Zadaniem \emph{SimulationRunnerService}u jest uruchomienie wyznaczonej symulacji z odpowiednimi parametrami oraz późniejsze zapisanie wyników. Serwis ten funkcjonuje jako osobny \emph{\docker Image}, który jest uruchamiany jako kontener przez \texttt{IDockerContainerManager}. Całość komunikacji pomiędzy poszczególną instancją i resztą systemu odbywa się przez system plików (więcej informacji na ten temat znajduje się w sekcji \ref{sec:dockerContainerManager}).

\par \emph{SimulationStandard} definiuje strukturę jaką powinna spełniać symulacja przygotowana przez użytkownika, jak i jej obsługiwane typy danych parametrów oraz rezultatów. Temu tematowi poświęcona jest sekcja \ref{sec:simulationStandard}.

\subsection{Zastosowane Wzorce Projektowe}

\subsection{Simulation Standard}
\label{sec:simulationStandard}

\subsection{Docker Container Manager}
\label{sec:dockerContainerManager}

\subsection{Docker Watch Service}
\label{sec:dockerWatchService}

\subsection{Schemat Bazy Danych}

\section{Wymagania i Konfiguracja}
\label{sec:wymaganiaIKonfiguracja}

\section{Endpoints}

\section{Obsługa błędów}


\chapter{Podsumowanie}
\label{cha:podsumowanie}

\section{Możliwy dalszy rozwój}

\subsection{Wersjonowanie Simulation Standardu}

\subsection{Remote Docker Host} % TODO Easy job, just start one serivce one that host that would present you all files

% itd.
\appendix
\chapter{Wyniki Symulacji Wyspy Królików}
\label{app:wynikiSymulacjiWyspyKrolikow}

\par W ramach procesu testowania aplikacji postanowiono wykorzystać symulację "Wyspy Królików" (więcej informacji można znaleźć w sekcji \ref{subsec:rabbitIslandSimulation}). Symulacja ta zawiera następujące parametry wejściowe:

\begin{center}
	\texttt{
		\begin{tabular}{|c | c | c |}
			\hline
			Nazwa Parametru & Typ & Przyjęta Wartość \\
			\hline
			\hline
			RabbitsInitialPopulation & long & Zmienna w zależności od testu \\
			\hline
			RabbitsMinChildren & long & 0 \\
			\hline
			RabbitsMaxChildren & long & 6 \\
			\hline
			RabbitsPregnancyDuration & long & 2 \\
			\hline
			RabbitsLifeExpectancy & long & 6 \\
			\hline
			WolvesInitialPopulation & long & Zmienna w zależności od testu \\
			\hline
			WolvesMinChildren & long & 0 \\
			\hline
			WolvesMaxChildren & long & 6 \\
			\hline
			WolvesPregnancyDuration & long & 4 \\
			\hline
			WolvesLifeExpectancy & long & 14 \\
			\hline
			TimeRate & long & 3600 \\
			\hline
			DeathFromOldAge & boolean & true \\
			\hline
			MaxCreatures & long & 500 \\
			\hline
			FruitsPerDay & long & 60 \\
			\hline
			MapSize & long & 500 \\
			\hline
			MutationChance & double & 0 \\
			\hline
			MutationImpact & double & 0 \\
			\hline
			OffspringGenerationMethod & long & 0 \\
			\hline
			Timeout & long & 600 \\
			\hline
		\end{tabular}
	}
\end{center}

Wartymi szczególnej uwagi są tutaj:
\begin{itemize}
	\item \texttt{MaxCreatures} - określa maksymalną ilość stworzeń symulowanych w tym samym czasie.
	\item \texttt{TimeRate} - parametr ten decyduje o ile razy szybciej będą działy się wydarzenia w symulacji niż w rzeczywistości. W naszym przypadku przyjmuje on wartość \emph{3600}, co oznacza, że \emph{1} sekunda czasu rzeczywistego to \emph{1} godzina czasu symulowanego.
	\item \texttt{DeathFromOldAge} - jeżeli ustawiony na wartość \texttt{true}, to wartości \texttt{RabbitsLifeExpectancy} i \texttt{WolvesLifeExpectancy} są brane pod uwagę i kreatury umierają nie tylko w wyniku obrażeń, ale i ze starości.
	\item \texttt{Timeout} - określa po jakim czasie (wyrażonym w sekundach czasu rzeczywistego) symulacja ma zostać przerwana.
\end{itemize}

Natomiast za pomocą \texttt{RabbitsInitialPopulation} i \texttt{WolvesInitialPopulation} będziemy sterować początkowymi populacjami ssaków na wyspie. Symulacja kończy się w momencie wymarcia wszystkich kreatur lub przekroczeniu czasu \texttt{Timeout}.

\subsection{Test 1: Zerowa populacja królików}

\par W przypadku pierwszego testu, postanowiona sprawdzić czy założenia symulacji pokryją się z otrzymanymi wynikami. Przy zerowej populacji królików, wszystkie wilki powinny po pewnym czasie wymrzeć z głodu. Sukces tego testu możemy zaobserwować na grafie \ref{fig:rabbitIslandTest1Diagram1}, który powstał na bazie otrzymanych wyników. Początkową populację ustawiono na \emph{14}.

\begin{figure}
	\includegraphics{RabbitIsland - Test 1}
	\label{fig:rabbitIslandTest1Diagram1}
	\caption{Wyspa Królików - Test 1 Diagram}
	\source{Opracowanie własne.}
\end{figure}

\subsection{Test 2: Zerowa populacja wilków}

\par Drugi test sprawdza sytuację analogiczną do pierwszej, tym razem jednak z królikami w roli głównej. Ilość stworzeń na start ustawiono na \emph{14}. Przewidywane są dwa poprawne rezultaty:
\begin{itemize}
	\item Populacja królików urośnie do takiego stopnia, że przychód nowych owoców nie będzie w staniej jej utrzymać. Następnie zacznie się nagłe wymieranie, spowodowane niedostatkiem pokarmu.
	\item Zostanie osiągnięty pewien poziom, na którym populacja królików ustabilizuje się.
\end{itemize}

\begin{figure}
	\includegraphics{RabbitIsland - Test 2}
	\label{fig:rabbitIslandTest2Diagram1}
	\caption{Wyspa Królików - Test 2 Diagram}
	\source{Opracowanie własne.}
\end{figure}

\par Na bazie otrzymanego wyniku (grafika \ref{fig:rabbitIslandTest2Diagram1}), możemy stwierdzić, że spełnił się drugi z przewidywanych scenariuszy, jednak nie ze względu na naturalną stabilizację, ale twardy limit stworzeń, ustawiony przez jeden z parametrów. Okresem wzbudzającym zainteresowanie, jest pierwsze kilka dni, gdzie widzimy, jak gromadzi się nadmiar owoców. Ilość owoców osiągnęła swoje globalne maksimum na początku piątego dnia. Rezultaty tego dostatku, możemy zaobserwować w następnych dniach, gdzie występują największe przyrosty populacji królików (koreluje to z ustawionym czasem trwania ciąży - \emph{2 dni}).

\subsection{Test 3: Mieszanie wilków z królikami}

\par Ostatni z wykonanych testów miał za zadanie sprawdzić interakcję pomiędzy dwoma typami stworzeń. Początkowe populacje zostały ustawione na \emph{14} wilków i \emph{24} króliki. Wszystkie pozostałe parametry pozostały bez zmian.

\begin{figure}
	\includegraphics{RabbitIsland - Test 3 1}
	\label{fig:rabbitIslandTest3Diagram1}
	\caption{Wyspa Królików - Test 3 Diagram 1}
	\source{Opracowanie własne.}
\end{figure}

\begin{figure}
	\includegraphics{RabbitIsland - Test 3 2}
	\label{fig:rabbitIslandTest3Diagram2}
	\caption{Wyspa Królików - Test 3 Diagram 2}
	\source{Opracowanie własne.}
\end{figure}

\par Z otrzymanych rezultatów (grafiki \ref{fig:rabbitIslandTest3Diagram1} i \ref{fig:rabbitIslandTest3Diagram2}) możemy wywnioskować, że takie dobranie populacji było nietrafione. Ze względu na stosunkowo nieduży rozmiar mapy, jak i dużą ilość drapieżników populacja królików szybko zmalała do pojedynczego przedstawiciela tego gatunku. Co z kolei spowodowało iż owoce zaczęły się gromadzić, jak i doprowadziło finalnie populację wilków do wymarcia z powodu głodu.


% \include{dodatekB}
% itd.

\printbibliography

\end{document}
