\chapter{Cel i motywacja}
\label{cha:celIMotywacja}

\par Zaprojektowanie i stworzenie modelu symulacji to tylko pierwszy krok. W zależności od natury konkretnego modelu, należy znaleźć odpowiednie wartości parametrów albo od razu uruchomić system z wybranymi uprzednio parametrami. Dopiero po tym procesie można przejść do głównego celu, dla którego w większości zaczął powstawać dany projekt, czyli do wnioskowania. Złudnym mogłoby być wrażenie zakończenia prac na etapie finalizacji implementacji. To właśnie testowanie i prawidłowe działanie symulacji potrafi zająć najwięcej czasu. Często niestety tego czasu brakuje, szczególnie w kontekście podejmowania różnej aktywności. Nie chcemy, aby nasze główne narzędzie pracy było zajęte przez obliczenia, które mogą trwać godzinami, a w skrajnych przypadkach nawet całymi dniami.

\par W trakcie okresu studiowania wraz z koleżankami i kolegami z kierunku, zaimplementowaliśmy wiele różnych symulacji. W większości były to symulacje niedeterministyczne, na bazie których wnioskowanie jest możliwe dopiero po wielokrotnym ich uruchomieniu. W niektórych przypadkach oznaczałoby wiele godzin pracy naszych komputerów, na co niestety nie mogliśmy sobie pozwolić. Skorzystanie z zewnętrznego serwisu oferującego udostępnienie komputerów również wydawało się kłopotliwe w aranżacji, a nawet gdyby udałoby się nam uzyskać dostęp do takowych maszyn to pojawiłaby się potrzeba samodzielnej implementacji dodatkowego systemu kolejkowania zadań ich i gromadzenia rezultatów.

\par Właśnie w celu przeciwstawienia się tym trudnościom powstał pomysł na niniejszy projekt. Sprowadzenia procesu uruchomiania systemu z odpowiednimi parametrami, jak i kolekcjonowania oraz prezentowania danych uzyskanych w postaci rezultatów do czynności trywialnej, dzięki stworzonemu systemowi informatycznemu. Głównym celem jest uproszczenie tego procesu, jak i odciążenie komputera osobistego osoby pracującej nad danym modelem.