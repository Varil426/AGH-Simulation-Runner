\chapter{Podsumowanie}
\label{cha:podsumowanie}

\section{Rezultat}

\par W trakcie rozwoju projektu udało się stworzyć aplikację oraz standard dla symulacji. Oba te elementy spełniają postawione przed nimi założenia, a połączone razem zapewniają użytkownikom wygodny interfejs pozwalający na uruchomienie wielu symulacji na oddalonym serwerze, zwalniając tym samym ich komputery osobiste. Aplikacja jest w stanie obsługiwać różne konfiguracje parametrów, jak i rezultatów oraz odpowiednio je obsługiwać i przechowywać. Wysoki poziom bezpieczeństwa został uzyskany dzięki wykorzystaniu \emph{\docker}a i konteneryzacji kodu \emph{3\textsuperscript{rd}-party}.

\par Projekt ten ma jednak w sobie wciąż duży potencjał. Możliwości na rozwój jest wiele (kilka z nich została przedstawionych w następnej sekcji, to jest \ref{sec:mozliwyDalszyRozwoj}). Poruszenie tematów takich jak różnice pomiędzy różnymi systemami kolejkowania, obliczenia z wykorzystaniem systemów rozproszonych czy w końcu różnymi metodami wirtualizacji.

\section{Możliwy dalszy rozwój}
\label{sec:mozliwyDalszyRozwoj}

TODO % TODO

\subsection{Wersjonowanie Simulation Standardu}

\subsection{Remote Docker Host} % TODO Easy job, just start one serivce one that host that would present you all files
