\chapter{Podsumowanie}
\label{cha:podsumowanie}

\section{Rezultat}

\par W trakcie rozwoju projektu udało się stworzyć aplikację oraz standard dla symulacji. Oba te elementy spełniają postawione przed nimi założenia, a połączone razem zapewniają użytkownikom wygodny interfejs pozwalający na uruchomienie wielu symulacji na oddalonym serwerze, zwalniając tym samym ich komputery osobiste. Aplikacja jest w stanie obsługiwać różne konfiguracje parametrów, jak i rezultatów oraz odpowiednio je obsługiwać i przechowywać. Wysoki poziom bezpieczeństwa został uzyskany dzięki wykorzystaniu \emph{\docker}a i konteneryzacji kodu \emph{3\textsuperscript{rd}-party}.

\par Projekt ten ma jednak w sobie wciąż duży potencjał. Możliwości na rozwój jest wiele (kilka z nich została przedstawionych w następnej sekcji, to jest \ref{sec:mozliwyDalszyRozwoj}). Poruszenie tematów takich jak różnice pomiędzy różnymi systemami kolejkowania, obliczenia z wykorzystaniem systemów rozproszonych czy w końcu różnymi metodami wirtualizacji.

\section{Możliwy dalszy rozwój}
\label{sec:mozliwyDalszyRozwoj}

\subsection{Wersjonowanie Simulation Standardu}

\par W dalszym rozwoju zaprezentowanej w tej pracy aplikacji wymagany byłby system wersjonowania \emph{Simulation Standard}u. Główną zaletą takiego podejścia byłaby kompatybilność wsteczna\english{Backward Compatibility}. Jeżeli aplikacja zostałaby uruchomiona dla użytkowników, to niepożądanym rezultatem byłaby niemożliwość uruchomienia stworzonego wcześniej oprogramowania. Z dokładnie tego samego powodu różne wersje \emph{API} często działają równolegle.

\subsection{Remote Docker Host}

\par Innym dość oczywistym rozszerzeniem jest zastosowanie \emph{Docker Host}a, który działałby na innej maszynie. Otworzyłoby to dodatkowo drogę do uruchomienia takiego serwisu na większej ilości urządzeń, co z kolei pozwoliłoby na lepsze skalowanie się systemu, jak i na zwiększenie jego stabilności, poprzez mniejszą zależność od konkretnej maszyny. Możliwą implementacją byłoby uruchomianie specjalnego serwisu na każdym z urządzeń, który przygotowywałby odpowiednie pliki po ich otrzymaniu (na przykład za pośrednictwem protokołu \texttt{HTTPS}) lub pobierał je bezpośrednio z bazy danych.

\subsection{Udostępnienie Simulation Standardu jako pakiet NuGet}

\par Jako iż przygotowany standard został upubliczniony na \emph{GitHub}ie w formie \emph{Open Source}, to nie istnieją żadne przeciwwskazania przed udostępnieniem go jako pakiet \emph{NuGet}\footnote{Menadżer pakietów używany w środowisku \emph{\dotnet{}}}. Ułatwiłoby to dostęp do najnowszej wersji (jak i poprzednich) dla użytkowników, jak i zapewniło prosty mechanizm do aktualizowania wersji standardu używanej w projektach.

\subsection{Aplikacja Frontendowa}

\par Aby w pełni spełnić założenia projektu, czyli w jak największym stopniu usprawnić kolejkowanie i uruchamianie symulacji, należałoby stworzyć aplikację \emph{frontend}ową. Byłaby ona użyteczna głównie dla użytkowników, którzy chcą dopiero zacząć swoją przygodę z \emph{Simulation Stadnaerd}em, jak i dla osób chcących przeprowadzać proste badania. Aplikacja ta mogłaby oferować dodatkowe formy prezentacji wyników, takie jak na przykład możliwość wyświetlania danych w postaci grafów. Należy oczywiście pamiętać, że nie istniałby żaden powód aby ograniczać możliwość korzystania z \emph{API}.

\subsection{Obsługa symulacji składających się z wielu Assembly}

\par Jak można zaobserwować na podstawie aplikacji, która powstała w ramach realizacji tej pracy dyplomowej, projekty tworzone w \emph{\dotnet{}} potrafią się rozrastać, a rezultatem ich kompilacji (przy ustawieniach domyślnych) może być więcej niż jeden plik \texttt{.DLL}. Jest to zarówno wada, jak i zaleta. Podział programu pozwala zastosować strategię \emph{Divide \& Conquer}, ale i wielokrotnie wykorzystać kod. Niestety kreuje to konieczność dodatkowej implementacja obsługi symulacji składających się z wielu \emph{assembly}. Możliwym rozwiązaniem tego problemu jest dodanie obsługi dla archiwów \emph{ZIP} i przeszukiwanie ich (lub wymaganie od użytkownika, aby wskazał odpowiedni plik, który służyłby jako \emph{Entry Point}).