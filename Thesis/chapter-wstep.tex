\chapter{Wstęp}
\label{cha:wstep}
% TODO Add some bibliography regarding Moore's Law
\par Symulacje komputerowe służą nam do rozmaitych celów. Od prostych symulacji mających na celu przedstawienie działania jakiegoś zjawiska do naprawdę zaawansowanych, wieloagentowych systemów, mających na celu przewidzenie zachowania tłumu podczas opuszczania obiektu sportowego po wieczornym meczu siatkówki. Informacje uzyskiwane w ten sposób pomagają inżynierom na całym świecie projektować lepsze systemy, znajdować niedopatrzenia w już istniejących, jak i je optymalizować. Jednak aby osiągnąć te cele potrzebna jest znacząca moc obliczeniowa, którą często nie dysponujemy, używając komputerów osobistych. Nawet jeżeli szybkość z jaką urządzenia, które znajdują się pod lub na naszych biurkach, potrafią dokonywać obliczeń rosła przez ostatnie dziesięciolecia zgodnie z Prawem Moore'a, to tak samo rosły nasze oczekiwania wobec nich. Kiedy dziś mówimy o optymalizacji procesów, symulowaniu zjawiska fizycznego czy wspomnianej wcześniej symulacji środowiska składającego się z wielu agentów, to chcemy przeprowadzić nie jeden czy dwa eksperymenty, ale setki lub nawet tysiące.

\par Problem ten pozwala nam rozwiązać tak zwany \emph{Cloud Computing}. Oferuje on możliwość oddelegowania obliczeń do specjalnie przeznaczonego serwera (bądź też serwerów). Takie rozwiązania istnieją już w sieci, jednak jeżeli zaczniemy się im przyglądać, to dostrzeżemy, że większość z nich skłania się ku symulacjom konkretnych zjawisk fizycznych po dostarczeniu pliku w odpowiednim formacie. Jest to dość istotne ograniczenie, szczególnie kiedy interesują nas symulacje innej natury. Oczywiście trzeba wspomnieć również o superkomputerach, których reprezentantem jest między innymi znajdujący się w Cyfronecie AGH \emph{\prometheusAgh{}}. Maszyny te są jednak trudno dostępne. Uruchomienie na nich jakikolwiek obliczeń wymaga zazwyczaj otrzymania specjalnej zgody (grantu w przypadku \emph{\prometheusAgh{}}a), jak i samodzielnego przygotowania pewnego rodzaju systemu, który uruchamiałby automatycznie przygotowaną przez na symulację.

\par W mojej pracy inżynierskiej poruszam przedstawione powyżej problemy oraz proponuję ich rozwiązanie.

%---------------------------------------------------------------------------

\section{Zakres pracy}
\label{sec:zakresPracy}

Zakresem pracy jest:
\begin{itemize}
	\item Przygotowanie standardu do implementacji symulacji.
	\item Stworzenie aplikacji webowej, spełniającej następujące warunki:
	      \begin{itemize}
		      \item Istnienie interfejsu, dzięki któremu możliwa jest obsługa zarówno przez użytkownika, jak i przez aplikacje \emph{3\textsuperscript{rd}-party}.
		      \item Istnienie systemu obsługi błędów.
		      \item Obsługa wspomnianego wcześniej standardu symulacji (opisanego dokładniej w sekcji \ref{sec:simulationStandard}).
		      \item Możliwość bezpiecznego uruchomienia kodu \emph{3\textsuperscript{rd}-party}.
		      \item Przechowywanie informacji o użytkowniku, plikach przez niego dostarczonych, jak i parametrach i rezultatach przeprowadzonych symulacji.
	      \end{itemize}
	\item Przetestowanie działania stworzonego systemu.
\end{itemize}

%---------------------------------------------------------------------------

\section{Zawartość pracy}
\label{sec:zawartoscPracy}

Zawartością pracy jest:
\begin{itemize}
	\item Określenie celu i motywacji projektu.
	\item Uchwycenie problemu od strony technicznej, wskazanie miejsc, które mogą okazać się trudne w implementacji.
	\item Opisanie zastosowanych technologii oraz wzorców projektowych.
	\item Prezentacja oraz wyjaśnienie działania aplikacji powstałej w toku pracy nad projektem.
	\item Sformułowanie podsumowania zawierającego uzyskane rezultaty oraz pomysły na dalszy rozwój stworzonego systemu.
\end{itemize}